% Created 2021-08-23 Mon 17:56
% Intended LaTeX compiler: pdflatex
\documentclass[a4paper,11pt]{article}
\usepackage[utf8]{inputenc}
\usepackage[T1]{fontenc}
\usepackage{fixltx2e}
\usepackage{graphicx}
\usepackage{longtable}
\usepackage{float}
\usepackage{wrapfig}
\usepackage{rotating}
\usepackage[normalem]{ulem}
\usepackage{amsmath}
\usepackage{textcomp}
\usepackage{marvosym}
\usepackage{wasysym}
\usepackage{amssymb}
\usepackage{hyperref}
\usepackage{mathpazo}
\usepackage{color}
\usepackage{enumerate}
\definecolor{bg}{rgb}{0.95,0.95,0.95}
\tolerance=1000
            \usepackage[table]{xcolor}
\usepackage[margin=0.9in,bmargin=1.0in,tmargin=1.0in]{geometry}
\usepackage{algorithm2e}
\usepackage{algorithm}
\usepackage{amsmath}
\usepackage{arydshln}
\usepackage{subcaption}
\newcommand{\point}[1]{\noindent \textbf{#1}}
\usepackage{hyperref}
\usepackage{csquotes}
\usepackage{graphicx}
\usepackage{bm}
\usepackage{subfig}
\usepackage[mla]{ellipsis}
\parindent = 0em
\setlength\parskip{.5\baselineskip}
\usepackage{pgf}
\usepackage{tikz}
\usetikzlibrary{shapes,arrows,automata,quotes}
\usepackage[latin1]{inputenc}
\usepackage{adjustbox}

\linespread{1.1}
\hypersetup{pdfborder=0 0 0}
\author{Gene Ting-Chun Kao}
\date{2021-08-23}
\title{How to write in org-mode and export to pdf\\\medskip
\large simple template}
\begin{document}

\maketitle



\section*{Section title}
\label{sec:org8a848e8}

This is a placeholder for writing contents


\subsection*{Image}
\label{sec:orga7f7f86}

This is an how we can refer to an image, see figure \ref{fig:org57c7ce9}.

\begin{figure}[htbp]
\centering
\includegraphics[width=0.4\textwidth]{./images/Leopard-ICON-circle.png}
\caption{\label{fig:org57c7ce9}Leopard icon}
\end{figure}

There are other ways of showing sub-images and display sub-captions like using in latex,
see figure \ref{org42da5b8}

\begin{figure}
    \centering
    \subfloat[label 1]{{\includegraphics[width=0.3\textwidth]{./images/Leopard-ICON-circle.png} }}
    \subfloat[label 2]{{\includegraphics[width=0.3\textwidth]{./images/Leopard-ICON-circle.png} }}
\caption{\label{org42da5b8}figures with captions}
\end{figure}


\subsection*{Table}
\label{sec:org5f6b2bf}

\begin{center}
\begin{tabular}{llr}
Author & Email & Institution-ID\\
\hline
Gene Ting-Chun Kao & your.email@email.edu & 1\\
Your name &  & 2\\
another name &  & 3\\
\end{tabular}
\end{center}



\section*{Section title}
\label{sec:org7c0a0fb}

\subsection*{Mathematics in latex}
\label{sec:org1d8c8fa}

Check equation \ref{eq:org616698d}.

\begin{equation}
\label{eq:org616698d}
f(x) = {s_0} = \frac{{\sum\limits_i {n_i^T(x - {x_i}){\Phi _i}(x)} }}{{\sum\limits_i {{\Phi _i}(x)} }}
\end{equation}

To have a set of equations and to align them:

\begin{equation}
\begin{array}{l}
\begin{split}
\max \;\;\; {{\bf{c}}^T}{\bf{x}} & \\
s.t. \;\;\; {\bf{Ax}} & \le {\bf{b}} \\
{\bf{x}} & \ge {\bf{0}}
\end{split}
\end{array}
\end{equation}

\subsection*{Graph}
\label{sec:org03388db}

Check out the graph in figure \ref{org8d3df96}.

\begin{figure}[H]
\caption{Max flow min cut, max flow = 19}
\vspace*{5mm}
\centering
\begin{tikzpicture}[->,>=stealth',shorten >=1pt,auto,node distance=2.8cm,
                    semithick,
xs/.style = {xshift=#1 mm},
ys/.style = {yshift=#1 mm},
every edge quotes/.style = {auto, pos=0.5, % <-- =.3?
                            inner sep=2pt, font=\footnotesize}
                        ]
  \tikzstyle{every state}=[fill=black,draw=none,text=white]

  \node[state]         (A)                    {$Source$};
  \node[state]         (B) [above right of=A] {$n_0$};
  \node[state]         (C) [right of=A]       {$n_1$};
  \node[state]         (D) [below right of=A] {$n_2$};
  \node[state]         (E) [right of=C]       {$Sink$};

  \path (A) edge                            node {9} (B)
            edge [color=blue]               node {7} (C)
            edge [color=blue]               node {5} (D)
        (B) edge [color=blue]               node {4} (E)
            edge [color=blue, bend left=10] node {3} (C)
        (C) edge [bend left=10]             node {5} (D)
            edge [bend left=10]             node {2} (B)
            edge                            node {7} (E)
        (D) edge [bend left=10]             node {1} (C)
            edge                            node {8} (E);
\draw[rounded corners=10mm, red, densely dashed]
    ( [xs=-10] D.west)  -- ( [xs=-10] C.west) -- ( [ys=10] C.north) -- ([ys=20] E.north);

\end{tikzpicture}
\label{org8d3df96}
\end{figure}


\subsection*{Algorithm}
\label{sec:org303a68b}

\begin{algorithm}[H]
\SetAlgoLined
 \KwData{Initial bounding-box $Q_0$ for $\Theta$, $QBest = Q_0$, $delta = 3$, stack $\Omega  = \{ {Q_0}\}$ }
 \KwResult{Optimal Q^* = QBest \in \Omega }
 \While{U_k - L_k > 1}{
           Pop $Q_k \in \Omega$  \\
           Prune $\Omega$ if current node is impossible solution node \\
           Compare $L_k$ from $Q_k$ and $QBest$ \\
           \If{$Q_{k}.L_k > QBest.L_k$}{
                     $QBest = Q_k$
           }
           Split $Q$ into $Q_I$ and $Q_{II}$ \\
           Find best condidate from $Q_I$ and $Q_{II}$ and add them to stack $\Omega$
 }
 \caption{How to write algorithms}
\label{org651b6bd}
\end{algorithm}

\subsection*{Flowchart}
\label{sec:org14a3efe}

\tikzstyle{decision} = [diamond, draw, fill=blue!20, 
    text width=4.5em, text badly centered, node distance=3cm, inner sep=0pt]
\tikzstyle{block} = [rectangle, draw, 
    text width=5em, text centered, rounded corners, minimum height=4em]
\tikzstyle{line} = [draw, -latex']
\tikzstyle{cloud} = [draw, ellipse, node distance=3cm, fill=red!20, 
    minimum height=2em]
\tikzstyle{term} = [draw,
    rounded rectangle, 
    minimum width=2.5cm,minimum height=1cm]
\tikzstyle{io} = [draw,
    trapezium,
    trapezium left angle = 65,
    trapezium right angle = 115,
    trapezium stretches,
    text width=5em, text centered]

This flowchart in Fig. \ref{org1a5bba3} is modified from \href{https://texample.net/tikz/examples/simple-flow-chart/}{this latex code}. 

\begin{figure}[htbp]
\begin{adjustbox}{max width=\columnwidth,center}

\begin{tikzpicture}[node distance = 2cm, auto]

\node [term] (start) {start};
\node [io, below of=start] (input) {input i/o};
\node [block, below of=input] (init) {initialize model};
\node [cloud, left of=init] (expert) {expert};
\node [cloud, right of=init] (system) {system};
\node [block, below of=init] (identify) {identify candidate models};
\node [block, below of=identify] (evaluate) {evaluate candidate models};
\node [block, left of=evaluate, node distance=3cm] (update) {update model};
\node [decision, below of=evaluate] (decide) {is best candidate better?};
\node [term, below of=decide, node distance=3cm] (stop) {stop};

\path [line] (start) -- (input);
\path [line] (input) -- (init);
\path [line] (init) -- (identify);
\path [line] (identify) -- (evaluate);
\path [line] (evaluate) -- (decide);
\path [line] (decide) -| node [near start] {yes} (update);
\path [line] (update) |- (identify);
\path [line] (decide) -- node {no}(stop);
\path [line,dashed] (expert) -- (init);
\path [line,dashed] (system) -- (init);
\path [line,dashed] (system) |- (evaluate);

\end{tikzpicture}
\end{adjustbox}
\caption{This is my flow chart}
\label{org1a5bba3}
\end{figure}

\subsection*{Citation}
\label{sec:org81fd0c4}


This is how we can cite paper \cite{kao2017assembly}

\bibliographystyle{plain}
\bibliography{org-bib-refs}
\end{document}
